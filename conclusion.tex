\section{Conclusion}
\label{sec:conclusion}

The main fact presented in this book chapter is that we have successfully
integrated Catalyst into \CS (a computational fluid dynamics code
developed at EDF R\&D). Both Catalyst and \CS are Open Source software
and this development can be download, used or tested freely by everyone. After
testing the prototype in our corporate supercomputer Ivanhoe, we find Catalyst to
be a relevant solution to provide \CS users with visualization
co-processing. Catalyst proved to allow a simple and fast implementation of an
adaptor.

The presented results are based on a 51M and a 204M elements mesh, which is
above the average size case used by EDF engineers in our industrial environment.
We plan to perform simulations on at least 400M elements meshes in the near
future, using the same supercomputer. We have also performed simulations up to
300 nodes and are currently planning not using more nodes. This is due to the
typical simulation node size in Ivanhoe being around 150 nodes for our
engineers. We also plan to work on another of our corporate supercomputers, an
IBM BG/Q with 65k cores. In that case, we will test on a much larger number of
cores.

The increase of memory use, described in the results section, indicates that
memory optimizations are to be performed before running on the IBM BG/Q. We did
not, in this study, perform any delicate memory tweaking in order to reduce the
memory consumption. We are currently working on this point, experimenting with
the new VTK in-situ data structures implemented recently by Kiware, the
so-called ``zero copy VT''. This approach aims to facilitate the memory
management in the adaptor without the use of complicated pointer manipulation;
we expect to reduce memory overhead without much increasing code complexity.

Another ongoing development consists on how we deal with the ghost levels
generated by \CS. Indeed, we want to use the same spatial partition of
the meshes for \CS and Catalyst, the aim being not to degrade the
execution time by “not necessary data exchanges” among MPI ranks. We currently
use ParaView D3 filter (a filter originally performing a redistribution of the
data among MPI processes) as a ghost cell handler. However, we asked Kitware for
the integration in ParaView/Catalyst of a new filter to perform a direct
generation of ghost cells from existing distributed data. This development has
been finished in december 2013 before this book chapter is published.

This chapter has been dedicated on how to deal with large data using visual
co-processing but we are also testing the computational-steering capabilities of
Catalyst, the so-called Live Catalyst. This currently allows the modification of
the ParaView pipeline parameters while the numerical simulation is running.

In conclusion, we are mostly satisfied with the integration of Catalyst in
\CS. The first version of our integration will be released as part of a
new version of this open-source software.



