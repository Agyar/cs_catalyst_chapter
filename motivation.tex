\section{Motivation}
\label{sec:motivation}

Most numerical simulation engineers at EDF R\&D are currently visualizing lower
temporal and spatial resolution versions of their simulations, in order to avoid
I/O issues and cumbersome visualisation procedures, when large quantities of
data are involved. We believe other industries dealing with large simulations
are having the same problem. This is the reason why we decided to address the
subject of coprocessing and in-situ visualization. Our aim was to provide our
engineers with an operational research-oriented tool in a mid-term basis. For
this, we have evaluated Catalyst as an industrial tool for performing In-Situ
visualization. We believe that the experiences we found, positive and negative,
are illuminating for the community, especially for the simulation teams
(including simulation code developers and the visualization experts they work
with) that are considering the transition to in situ processing.

First of all, we believe that it is important to better describe the scope of
our industrial visualisation solutions before in-situ processing was being
tested. In table XXX we show the results of a simple subjective experiment
conducted by one of our engineers. During 2012, she meshed a simple cube at
different resolutions, using the SALOME platform, and then tried to visualise
the results giving a subjective evaluation of how she could work. She used an
EDF R\&D standard scientific PC that contains 8 Gb of RAM. The visualisation
system of SALOME is an integration of ParaView. Table XXX clearly shows that she
started working without an immediate system response for meshes which contain
more that 10 Millions cells and for 50 Million cells the system was not
responding. At the time of this test was done some of our R\&D engineers were
already performing simulations with meshes of around 200 Millions cells and less
than two years afterwards with 400 Millions cells. This implies that coping the
results from the computer to their scientific stations is not possible because
afterwards they will be block in their visualisation or other post-processing
tasks. This is a serious bottleneck in an industrial system that motivated the
beginning of this work.

\begin{table}
\centering
\begin{tabular}{|p{1.5cm}|p{2.5cm}|p{2.50cm}|p{1.50cm}|p{1.50cm}|p{1.50cm}|}
\hline
\multicolumn{6}{|c|}{\textbf{MESH SIZE MANIPULATION EXPERIMENT}}\\
\hline
Number of cells & 10 Thousands & 100 Thousands & 1 Millions & 10 Millions & 50 Millions \\
\hline
RAM(\%) & <50\% & <50\% & <50\% & 100\% & Saturated \\
\hline
Reaction time & Inmediate & Inmediate & 2 to 3 seconds & Unconfortable & Not responding \\
%\hline
 %& & & \\
%$CASE$\_$A$ & 51M hexahedrals, \newline industrial size case & \textbf{heavy}:
%\newline volume rendering, \newline celldatatopointdata \newline and glyphs  &
%5a 5c 5e\\
\hline
\end{tabular}
%\vspace{-0.1in}
\caption{Subjective characterization of the reaction time of the SALOME platform for different mesh sizes.}
\label{fig:tab}
%\end{figure}
\vspace{-0.15in}
\end{table}


1) The user generates a ``light version'' of the mesh. This step has already been
discussed in section~\ref{sec:pip_conf_tools}. Indeed, the user possess a CAD (Computer Aided
Design) version of the geometry that is parametrized, it is then possible to
obtain meshes at different spatial resolutions. A ``light mesh'' of small size in
memory and representative of the CAD geometry is obtained. Figure~\ref{fig:piece} represents
the ``light version'' of the mesh used in our experiments.


3) The mesh and the fields obtained at the end of step 2 are read in ParaView
and the user can define her/his visualisation pipeline. At the end of this step
a simple click in the ParaView interface will create a Python file that
programmatically defines the visualisation operations that will be performed
$in$-$situ$.






