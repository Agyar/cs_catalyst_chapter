\section{Related Work}
\label{sec:related}

The size of generated data has become an important subject in high performance 
computing, due to the need of a better I/O efficiency in our computing 
system. To answer this problem, several visualization systems have been created.
We can distinguish two main approaches in recent solutions. The first one is to 
integrate a specific $in$-$situ$ visualization directly to the simulation code. 
Such approach proved to be an efficient way to provide co-processing for a given
simulation as well as a visualization system as it is the case in the hurricane
prediction~\cite{4015457} and earthquake simulation~\cite{4090186} systems.
This method has been proven to lead to good performances but is limited 
to a specific implementation, which does not apply directly to our needs.

The second approach is to provide a general post-processing framework letting the
simulation and the visualization code communicate together. EPSN which is a
general coupling system, allows for the connection of M simulation nodes to N
visualization nodes through a network~\cite{4020782}. This solution is a
loosely coupled approach, requiring separate resources and data transfer
through the network. This approach presents the advantage of not overloading
the nodes used for computation. Thus the visualization code does not interfere
with the execution of the simulation. Based on the same approach, a ParaView
plug-in named ICARUS~\cite{6152102} has been developed. 
It differs from EPSN in design by having lower requirements as it only needs
the use of a single HDF5 library and file driver extension. Such solutions
offer tools for researchers to interact with their simulations by allowing
them, not only to monitor their current states but also to modify
the parameters of the remaining simulation steps. Those computational steering
solutions as well as the RealityGrid
project~\cite{Harting03computationalsteering} focus on interactivity with
simulation whereas our main objective is to provide $in$-$situ$ visualization
operations to researchers while minimizing I/O overhead and disk
space use. 

Both built upon the well known parallel visualization library VTK, 
the application frameworks VisIt~\cite{1532795} and ParaView~\cite{964413} both provide through
the possibility to co-process simulation data via libsim~\cite{2386230} and Catalyst~\cite{6092322} respectively.  
Those $in$-$situ$ solutions are tightly coupled and while they
limit potential interactions with the running simulation, they also highly
reduce the need of network data transfer. Thus, they contribute to circumventing
the inefficiency of high performance computing I/O systems.
Those solutions take their benefits from directly accessing the simulation memory to
perform visualization tasks by simply asking for a pointer to the available
data. One major drawback of this approach is the necessity to provide a coherent data 
layout to those libraries. Moreover, as this type of
solution often gains from computing pre-determined visualization tasks, it is
not well suited for results exploration.  As building a steering solution for Code\_Saturne is out of
the scope of this case study, we do not consider these drawbacks as a limitation. 

After evaluating the performance solutions offered by ParaView and VTK, we choose Catalyst as 
our co-processing library for our case study as it answers EDF's visualization 
needs while focusing on the  reduce of data amount use. Further, Kitware is still actively developing 
Catalyst, and we are optimistic that more services allowing the interactions
with the running simulation will soon be available.

