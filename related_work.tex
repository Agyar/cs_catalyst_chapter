\section{Related Work}
\label{sec:related}

The size of generated data has become an important subject in high performance 
computing, due to the need of a better input/output efficiency in our computing 
system. To answer this problem, several visualization systems has been created.
We can distinguish two main approaches in recent solutions. The first one is to 
integrate a specific $in$-$situ$ visualization directly to our simulation code. 
Such an approach proved to be an efficient way to provide coprocessing for a given
simulation plus visualization system as it is the case in the hurricane
prediction~\cite{4015457} and earthquake~\cite{4090186} simulation systems.
This method has been proved to lead to good performances but is limited 
to a specific implementation. Thus it does not respond to our needs. 

The second approach is to provide a general postprocessing framework letting the
simulation and the visualization code communicate together. EPSN which is a
general coupling system, allows for the connection of M simulation nodes to N
visualization nodes through a network~\cite{4020782}. This solution is a
loosely coupled approach, requiring separate resources and data transfer
through the network. This approach presents the advantage of not overloading
the nodes used for computation. Thus the visualization code does not interfere
with the execution of the simulation. Based on the same approach, a ParaView
plugin named ICARUS~\cite{6152102}
exists. It differs from EPSN in design by lower requirements as it only needs
the use of a single HDF5 library and file driver extension. Such solutions
offer tools for researchers to interact with their simulations by allowing
them, first to monitor their current states but also to modify
some parameters of the remaining simulation steps. Those computational steering
solutions as well as the RealityGrid
project~\cite{Harting03computationalsteering} focus on interactivity with
simulation whereas our main objective is to provide $in$-$situ$ visualization
operations to researchers while minimizing input/output overhead and disk
space use. 

Both built upon the well known parallel visualization algorithms
library VTK, VisIt~\cite{1532795} and ParaView~\cite{964413} provide through
libsim~\cite{2386230} and Catalyst~\cite{6092322} the possibility to coprocess
simulation data.  Those $in$-$situ$ solutions are tightly coupled and while they
limit potential interactions with the running simulation, they also highly
reduce the need of network data transfer. Thus, it contributes at circumventing
the inefficiency of high performance computing input/output systems.
Those solutions takes their benefits from directly accessing the simulation memory to
perform visualization treatments by simply asking a pointer to the available
data. One major drawback of this approach is the necessity to provide an
understandable data layout to those libraries. Moreover, as this type of
solution often gains from computing pre-determined visualization tasks, it is
not suited for results exploration.  As building a steering solution for Code\_Saturne is out of
the scope of this case study, we do not consider these drawbacks as a limitation. 

After evaluating the performances offered by Kitware~\cite{6092322}, we choose Catalyst as our coprocessing library for our 
case study as it answers our visualization needs while focusing on the 
reduce of data amount use. Ultimately, Kitware is still actively developing 
Catalyst, and we are optimistic that more services allowing the interactions
with the running simulation will soon be available.

