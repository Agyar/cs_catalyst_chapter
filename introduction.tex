\section{Introduction}
Computational Fluid Dynamics (CFD) is a fundamental step for the study and
optimization of electricity production. Indeed, current power plants use water
as a mean of convective heat transfer. Consequently, the simulation and
visualization of fluid dynamics phenomena is of great importance for the energy
industry. Electricité de France (EDF), one of the largest electricity producer in Europe, has
been developing for the past 15 years an open source CFD code named \CS. \CS performs
CFD computations on very large models~\cite{5644955}. EDF owns
several supercomputers that regularly run this code in order to perform CFD
analysis involving large amounts of data. In this context, the post-processing
and visualization steps become critical. 
%% =======

EDF also develops, in collaboration with OpenCascade and the French Center of
Atomic Research (CEA), an open-source numerical simulation platform called
SALOME. This platform provides generic methods for pre- and post-processing of
numerical simulations. SALOME is based on an open architecture made of reusable
components such as computer-aided design (CAD),
meshing, high performance computing (HPC) execution management, multi-physics coupling, data post-processing
and visualization. The visualization module of the SALOME platform is currently based
on the open-source post-processing platform ParaView. Furthermore, \CS is often used 
in conjunction with the the SALOME platform.

In the past, studies and improvements in scientific simulation have been mainly
focused on the solver, due to being the most cycle-consuming part in the
simulation process. Thus, visualization has been traditionally run sequentially
on a smaller computer and at the very end of the solver computation. At the
time, this was easily explained by the small need for both memory and
computation resources in most of the visualization cases. Nevertheless, with the
increase of our computational capabilities, we tend to use and generate much
more data than what we were used to. Thus, as the scale of CFD simulation
problems is getting wider, specific issues are emerging related to input/output
efficiency. In particular, data generated during the solver computation and
used for the visualization are the source of a worrisome overhead. Even worse,
some researchers are starting to spend more time writing and reading data
than actually running solvers and visualizations~\cite{1742-6596-125-1-012099}.
%Sometimes referred as Big Data, 
This new trend compels us to design new input/output (I/O) strategies and consider
visualization as a part of our high-performance simulation systems.

For some years, $in$-$situ$ visualisation techniques have been successfully 
applied in different contexts and mainly by research institutes. 
In this chapter, we present an overview of the efforts 
needed to transition a traditional simulation code to an $in$-$situ$ model 
in an industrial environment. This is the reason why care have been taken 
constructing uses cases that are representative of our current visualisation problems. 

Most fluid dynamic engineers at EDF R\&D are currently visualizing lower temporal and spatial 
resolution versions of their simulations in order to avoid I/O bottlenecks when large quantities of data are involved.
We decided to address the subject of co-processing and $in$-$situ$
visualization which has been proved to be an effective solution against the current
limitations of this problem~\cite{sandiareport}, \cite{4090186}. Our aim is to provide 
EDF engineers with an operational research-oriented tool in a mid-term basis.
%Moreover, while visualization has been traditionnally performed on smaller
%computer, studies show that visualization algorithms can often be run
%efficiently on recent supercomputers~\cite{4090186}. 
For this, we chose to evaluate Catalyst as an industrial tool for performing
$in$-$situ$ visualization. 
Catalyst is a library, developed by Kitware, which implements the
co-processing for ParaView by defining the visualization process through the ParaView user 
interface and exploiting VTK's parallel algorithms for the post-processing of data 
generated by numerical simulation~\cite{6092322}. 

In this chapter, we report a study upon the effectiveness and scalability of a
prototype implementation of the co-processing in an industrial case based on the
coupling of \CS with Catalyst. In section~\ref{sec:related} we
discuss related work on recent visualization $in$-$situ$ systems. We then
introduce, in section~\ref{sec:cs} \CS, the CFD code developed at EDF
R\&D. In section ~\ref{sec:catalyst} we present our integration of Catalyst into
\CS and how the system is used by EDF users in the context of fluid dynamic
simulations. Section~\ref{sec:results} describes our use case and presents
results on one of our corporate clusters. Finally, section~\ref{sec:conclusion} 
presents our analysis of the results and describes our ongoing and future work.
