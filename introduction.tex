\section{Introduction}
Computational Fluid Dynamics (CFD) is a fundamental step for the study and
optimization of electricity production. Indeed, current power plants use water
as a mean of convective heat transfer.  Consequently, the simulation and
visualization of fluid dynamics phenomena is of great importance for the energy
industry. Electricité de France (EDF), being one of the biggest electricity producer in Europe, has
developed for the past 15 years an open source CFD code named \CS,
allowing for the solution of very large models~\cite{5644955}. EDF has
several supercomputers that regularly run this code in order to perform
analysis involving large amounts of data. In this context, the postprocessing
and visualization of data becomes a critical point. 
%% =======

EDF also develops (in collaboration with OpenCascade and the French Center of
Atomic Research, CEA) an open-source numerical simulation platform called
SALOME. This platform provides generic methods for Pre- and Post-Processing of
numerical simulations. It is based on an architecture made of reusable
components. Among others, these components deal with: computer aided design,
meshing, HPC execution management, multi-physics coupling, data post-processing
and visualization. ParaView is currently integrated in this platform as a
visualization module. \CS is often used coupled to the SALOME platform.

In the past, studies and improvements in scientific simulation have been mainly
focused on the solver, due to being the most cycle-consuming part in the
simulation process. Thus, visualization has been traditionally run sequentially
on a smaller computer and at the very end of the solver computation. At the
time, this was easily explained by the small need for both memory and
computation resources in most of the visualization cases. Nevertheless, with the
increase of our computational capabilities, we tend to use and generate much
more data than what we were used to. Thus, as the scale of CFD simulation
problems is getting wider, specific issues are emerging related to input/output
efficiency. In particular, data generated during the solver computation and
used for the visualization are the source of a worrisome overhead. Even worse,
some researchers are starting to spend more time for writing and reading data
than for running solvers and visualizations~\cite{1742-6596-125-1-012099}.
%Sometimes referred as Big Data, 
This new trend is asking us to design new I/O strategies and consider
visualization as a part of our high-performance simulation systems.


Most fluid dynamic engineers at EDF R\&D are currently visualizing lower temporal and spatial 
resolution versions of their simulations in order to avoid I/O issues when large quantities of data are involved.
We decided to addres the subject of coprocessing and $in$-$situ$
visualization which has been proved to be an effective solution against the current
limitations of this problem~\cite{sandiareport}, \cite{4090186}. Our aim is to provide 
our engineers with an operational research-oriented tool in a mid-term basis.
%Moreover, while visualization has been traditionnally performed on smaller
%computer, studies show that visualization algorithms can often be run
%efficiently on recent supercomputers~\cite{4090186}. 
For this, we choose to evaluate Catalyst as an industrial tool for performing In-Situ visualization. 
Catalyst, developed by Kitware, is a library for Paraview that implements the
coprocessing, by defining the visualization process through Paraview and
exploiting the VTK's parallel algorithms for the processing of the simulation
data~\cite{6092322}. 

In this article, we propose a study upon the effectiveness and scalability of a
prototype implementation of the coprocessing in an industrial case based on the
coupling of \CS with Catalyst. In section~\ref{sec:related} we
discuss works related to recent visualization $in$-$situ$ systems. We will then
introduce in section~\ref{sec:cs} \CS, the CFD code developed at EDF
R\&D. In section ~\ref{sec:catalyst} we present our integration of Catalyst into
\CS and how it is used by the users in the framework of fluid dynamic
simulations. Section~\ref{sec:results} describes our use case and presents
results on one of our corporate clusters. Finally, section~\ref{sec:conclusion} 
presents our analysis of the results and describes our on-going and future work.
